\documentclass{article}
\usepackage{graphicx} % Required for inserting images
\usepackage{amsmath,amsthm,amsfonts}
\usepackage{geometry}
\geometry{a4paper, portrait, margin=2cm}

\title{Factorio Optimization Problem}
% \author{Brenton Candelaria}
% \date{September 2023}
\DeclareMathOperator*{\argmin}{argmin}
\DeclareMathOperator*{\argument}{arg}
\begin{document}

\maketitle

\section{Formalization}
\\

\indent Let $p \in \mathbb{N}^+$ be a natural number, $t \in \mathbb{Q}^+$ be a rational number, $m \in \mathbb{N}^+$ be a natural number, and $\vec{c} \in \mathbb{N}^n s.t. \forall i \in \mathbb{N}^+, i \leq n,0 < c_i $. Henceforth, $t$ is the ``period", $p$ is the ``output quantity", $m$ is the ``multiplicity", and $\vec{c}$ is the ``input". Then consider a directed acyclic graph $G$ defined as:
% (\exists j \in \mathbb{N}^+ \leq dim(\vec{c_b})\;s.t.\:  M(v_a)*P(v_a) \geq M(v_b) * c_j, j = C(v_b, v_a))\: \\
 %& \hspace{1.355in}\land
%, and } C(v_j, v_i) \text{ returns the index of } \vec{c} \text{ that } P(v_i) \text{ corresponds to as input to }, and } C(v_j, v_i) \text{ returns the index of } \vec{c} \text{ that } P(v_i) \text{ corresponds to as input to }
% & \hspace{1.355in}\\

\begin{align*}
G &=  \{V, E\}\\
 V &= \{v_k \: | \: v_k = (p_k, t_k, \vec{c}_k, m_k), \: k \in \mathbb{N}^+\}\\
 E &= \{e \: | \: E(v_a, v_b) \implies (\not \exists f \in \:E \: s.t.\: f = E(v_a, v_c), \: v_b \neq v_c)\}\\
 & \text{where } M(v_i) = m_i, P(v_i) = p_i \text{ for the multiplicities and output quantities of a vertex, respectively} \end{align*}

\\\medskip

Additionally, let a ``source vertex" be defined as:

\[
S \subset V = \{s \: | \: s = (p=1, t = 1, \vec{c} = \emptyset, m \in \mathbb{N}^+)\}
\]
\medskip

Finally, we let $T \subset V = V - S$, the set of vertices that are not sources (i.e.: their $\vec{c} \neq \emptyset$,) and $|T| = Q$. 

Notably, $Q = \sum\limits_{i=1}^n M(v_i)$ for $v_i \in T$ . Then, let the $\Delta$ operator denote the differences between two connected vertices, $v_i \text{ and } v_j$, as:
\begin{align*}
&\Delta_{i\rightarrow j} = \begin{cases}
     M(v_i)\frac{P(v_i)}{T(v_i)} - M(v_j)\frac{\vec{c}_{j,i}}{T(v_j)} & \text { iff } E(v_i, v_j)\\
     0 & \text{ otherwise}\\
\end{cases}\\
&\text{where } \vec{c}_{j,i} \text{ denotes the } i^{th} \text{ component of the vector } \vec{c} \text{ belonging to vertex } v_j,\\
&\text{and } T(v_j) \text{ denotes the period of some } v_j
\end{align*}

We overload the operator when it acts on set of vertices (as opposed to only two), $\Delta V$, to denote the same as the above across all such vertices and edges for a given graph, $G$, as:

\begin{align*}
&\Delta V = \sum\limits_{i=1}^{|V|-1} \sum\limits_{j=i+1}^{|V|} |\Delta_{i \rightarrow j}|\\
&\text{ for indices } i, j \text{ of some } v_i, v_j \in V\\
\end{align*}

%N.b.: $\Delta V \geq 0 \: \forall V$, by definition.
Then, let the optimizer $\mathcal{O}(Q, G): \mathbb{N} \times \{ V, E\} \rightarrow \vec{z} \in \mathbb{N}^{|V|}$ desire the following:

\begin{align*}
    \vec{z}& = \argmin_{\vec{z} \in \mathbb{N}^{|V|}}\Delta V\\
    & \text{ for } \\
    \Delta V & = \min_{\vec{z} \in \mathbb{N}^{|V|}} \sum\limits_{i = 1}^{|V|-1} \sum\limits_{j=i+1}^{|V|} \Delta_{i \rightarrow j}\\
    & \text{where } z_i = M(v_i) \: \forall v_i \in V\\
\end{align*}


% Nota bene: WITH THE ABOVE FORMALIZATION, Jared's "greedy' algorithm of minimizing "effective" rates *should* work now
% And all of these definitions *should* be exactly/approximately what the real problem is
% Nota bene 2: Bottlenecking *shouldn't* be a problem basically ever, esp. if we favor overproduction if absolutely necessary for the optimization
\end{document}