\documentclass{article}
\usepackage{graphicx} % Required for inserting images
\usepackage{amsmath,amsthm}
\usepackage{geometry}
\geometry{a4paper, portrait, margin=2cm}

\title{Factorio Optimization Problem}
% \author{Brenton Candelaria}
% \date{September 2023}
\DeclareMathOperator*{\argmin}{argmin}
\DeclareMathOperator*{\argument}{arg}
\begin{document}

\maketitle

\section{Formalization}
\\


\indent Let $p \in \mathbb{N}^+$ be a natural number, $r \in \mathbb{Q}^+$ be a rational number, $m \in \mathbb{N}^+$ be a natural number, and $\vec{c} \in \mathbb{N}^n s.t. \forall i \in \mathbb{N}^+, i \leq n, c_i \geq 0 $. Henceforth, $r$ is the ``rate", $p$ is the ``output quantity", $m$ is the ``multiplicity", and $\vec{c}$ is the ``input". Then consider a directed acyclic graph $G$ defined as
% (\exists j \in \mathbb{N}^+ \leq dim(\vec{c_b})\;s.t.\:  M(v_a)*P(v_a) \geq M(v_b) * c_j, j = C(v_b, v_a))\: \\
 %& \hspace{1.355in}\land
%, and } C(v_j, v_i) \text{ returns the index of } \vec{c} \text{ that } P(v_i) \text{ corresponds to as input to }, and } C(v_j, v_i) \text{ returns the index of } \vec{c} \text{ that } P(v_i) \text{ corresponds to as input to }
% & \hspace{1.355in}\\

\begin{align*}
G &=  \{V, E\}\\
 V &= \{v_k \: | \: (p_k, r_k, \vec{c}_k, m_k), \: k \in \mathbb{N}^+\}\\
 E &= \{e \: | \: E(v_a, v_b) \implies (\not \exists f \in \:E \: s.t.\: f = E(v_a, v_c), \: v_b \neq v_c)\}\\
 & \text{where } M(v_i) = m_i, P(v_i) = p_i \text{ for the multiplicities and output quantities of a vertex, respectively} \end{align*}

\\\medskip

Additionally, let a ``source vertex" be defined as:

\[
S \subset V = \{s \: | \: s = (p=1, r = 0, \vec{c} = \emptyset, m = 1)\}
\]
\medskip

Finally, we let $T \subset V = V - S$, the set of vertices that are not sources, and $|T| = Q$. 

Notably, $Q = \sum\limits_{i=1}^n M(v_i) \in \mathbb{N}^+$. Then, let the $\Delta$ operator denote the differences between two connected vertices, $v_i \text{ and } v_j$, as:
\begin{align*}
&\Delta_{i\rightarrow j} = M(v_i)*P(v_i) - M(v_j)*\vec{c}_{C(v_j, v_i)}\\
&\text{ where } C(v_j, v_i) \text{ returns the index of } v_j \text{ for the unique } v_i \text{ on the edge connecting the two.}
\end{align*}

We overload the operator for a vector representing multiplicities of indices, $\Delta \vec{z}$, to denote the same as the above across all such vertices and edges for a given graph, $G$, as:

\begin{align*}
&\Delta \vec{z} = \sum\limits_{i=0}^{|V|-1} \sum\limits_{j=i+1}^{|V|} \Delta_{i \rightarrow j}\\
&\text{ for index pairs } i, j, \in \vec{z}
\end{align*}


Then, let the optimizer $\mathbb{O}(Q, G): \mathbb{N} \times \{ V, E\} \rightarrow \vec{z} \in \mathbb{Z}^{|V|}$ desire the following:

\begin{align*}
    \vec{z}& = \argument_{\vec{z} \in \mathbb{N}^{|V|}}\Delta \vec{z}\\
    & \text{ for } \\
    0 & \leq \min_{\vec{z} \in \mathbb{Z}^{|V|}} \sum\limits_{i = 0}^{|V|-1} \sum\limits_{j=i+1}^{|V|} \Delta_{i \rightarrow j}\\
    & \text{where } z_i = M(v_i), i \in \mathbb{N}^+ \leq |V|\\
\end{align*}

\end{document}
